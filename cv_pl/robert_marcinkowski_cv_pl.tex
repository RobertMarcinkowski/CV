\documentclass[12pt,a4paper,sans]{moderncv}

%% ModernCV themes
\moderncvstyle{classic}
%\moderncvstyle{casual}
\moderncvcolor{blue}
\moderncvicons{awesome}
\renewcommand{\familydefault}{\sfdefault}
\nopagenumbers{}

%% Character encoding
\usepackage[utf8]{inputenc}
\usepackage{polski}
%% Adjust the page margins
\usepackage[scale=0.8]{geometry}

\setlength{\hintscolumnwidth}{3cm}

%% Personal data
\firstname{Robert}
\familyname{Marcinkowski}
\title{Curriculum~Vitae}
\address{os. Zamoyskiego 7/5}{62-020 Swarzędz} 
\mobile{698~034~854} 
%\phone{+2~(345)~678~901} 
%\fax{+3~(456)~789~012} 
\email{robertm1506@gmail.com} 
\homepage{robb15.usermd.net} 
%\linkedin{linkedin.com/in/robert-marcinkowski-b366a2a6} 
\social[github]{RobertMarcinkowski}
\social[linkedin]{robert-marcinkowski-rm} 
%\extrainfo{additional information} 
\photo[67pt][0.4pt]{robert_marcinkowski_cv.jpg}
%\quote{Some quote (optional)}

%%------------------------------------------------------------------------------
%% Content
%%------------------------------------------------------------------------------
\begin{document}
\vspace*{-3em}
\makecvtitle
%\pagestyle{empty}
\vspace*{-2.8em}
\lfoot{Wyrażam zgodę na przetwarzanie moich danych osobowych dla potrzeb niezbędnych do realizacji procesu rekrutacji (zgodnie z Ustawą z dnia 29.08.1997 roku o Ochronie Danych Osobowych; tekst jednolity: Dz. U. 2016 r. poz. 922).}
\section{Wykształcenie}
\cventry{10.2011--10.2016}{automatyka i robotyka}{Wydział Informatyki}{Politechnika Poznańska}{}{studia dzienne I i II stopnia, specjalizacja: \textbf{Reprogramowalne Systemy Sterowania}}

\cventry{19.10.2016}{Praca magisterska}{DR.ADAS: System wspomagania kierowcy z dynamiczną rekonfiguracją warstwy sprzętowej}{}{}{}

\cventry{10.02.2015}{Praca inżynierska}{Sterownik procesu przemysłowego zawierający komputer na module}{}{}{}

\section{Doświadzczenie zawodowe}
\cventry{08.2014--09.2014}{Automatyk}{Kompania Piwowarska S. A.}{Poznań}{praktyka studencka}{Zakres obowiązków: montaż układów pomiarowych, programowanie sterowników PLC}
% firmy Siemens
\cventry{04.2014}{Pracownik produkcji}{Omni3D sp. z o. o.}{Poznań}{praca dorywcza}{Zakres obowiązków: przygotowywanie elementów do montażu}

\section{Kursy}
\cventry{10.2016--obecnie}{Java}{Software Development Academy}{}{}{}

\section{Języki}
\cvitem{Język angielski}{C1, bardzo dobra znajomość w mowie i piśmie}
\cvitem{Język niemiecki}{A1, poziom podstawowy}

\section{Umiejętności i kompetencje osobiste}
\cvlistitem{umiejętność programowania w języku Java}
\cvlistitem{znajomość relacyjnych baz danych SQL}
\cvlistitem{umiejętność pracy z systemem kontroli wersji Git}
\cvlistitem{podstawowa znajomość: HTML, CSS, JavaScript, jQuery, Maven, VHDL, C}
\cvlistitem{znajomość podstaw elektroniki, automatyki, układów FPGA, ARM}
\cvlistitem{znajomość systemów Windows, Linux}
\cvlistitem{zdolność do szybkiej nauki, kreatywność}
\cvlistitem{kultura osobista, samodzielność, cierpliwość}

\section{Zainteresowania}
\cvlistdoubleitem{nowoczesne technologie}{pływanie}
\cvlistdoubleitem{filmy, seriale}{literatura fantasy}



\end{document}